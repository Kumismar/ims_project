\documentclass[a4paper, 11pt, hidelinks]{article}
\usepackage[left=2cm, top=3cm, text={17cm, 24cm}]{geometry}
\usepackage[czech]{babel}
\usepackage[utf8]{inputenc}
\usepackage{hyperref}
\usepackage{tabularx}
\usepackage{booktabs}
\usepackage{xcolor}
\usepackage{graphicx}
\usepackage{amsmath}
\usepackage{dirtree}

\bibliographystyle{plainurl}

\begin{document}

\begin{titlepage}
    \begin{center}
        \Huge \textsc{Vysoké učení technické v~Brně} \\
        \huge \textsc{Fakulta informačních technologií} \\
        \vspace{\stretch{0.382}}
        \Huge Negativní faktory ovlivňující pěstování pšenice
        \huge Simulační studie k projektu IMS\\
        \vspace{\stretch{0.618}}
        \Large \today{} \hfill Ondřej Koumar, Adam Malysák
    \end{center}
\end{titlepage}

\tableofcontents

\newpage

\section*{Úvod}\label{0_uvod}
\addcontentsline{toc}{section}{Úvod}

Pěstování a zpracování obilovin ovlivňují různé faktory.
Mohou to být například faktory spojené s~počasím, škůdci nebo člověkem.
Tato práce je zaměřena na sledování množství produkované mouky při pěstování pšenice.
V úvahu bere i futuristický/sci-fi faktor, který může roční produkci pšenice úplně zdevastovat.
Podmínky jsou simulovány v mírném klimatu, konkrétně v České republice.

Smyslem experimentů s implementovaným modelem je simulovat hmotnostní ztráty výsledného produktu\,--\,mouky
v závislosti na negativních faktorech, které produkci mohou ovlivnit.
Je zahrnuta i ztrátovost hmotnosti při mletí pšenice, předpokládá se mletí na mouku bez otrub.

Data obsažená v modelu vychází ze statistik Českého hydrometeorologického ústavu (ČHMÚ)\cite{CHMU} pro faktory spojené
s teplotou, ovlivňování růstu pšenice z jiných odborných článků, které budou citovány dále ve studii.

\section{Rozbor tématu a použitých metod/technologií}\label{1_rozbor}

Zkoumaný model bere v úvahu hypotetický farmářský podnik, který vlastní 200\,ha pole, na kterém se pěstuje výhradně pšenice.
Ke sklizni jsou k dispozici dvě auta na transport vymlácené pšenice z pole do skladu mlýna a jeden kombajn na samotné sklízení.
Auta nějakou dobu tráví na cestě do a ze skladu, nějaký čas také tráví v samotném skladě, kde se náklad vyloží a řidič se zde může zdržet.

Pěstování pšenice je rozděleno na několik fází.

V první fázi je modelování čekání zaseté pšenice přes zimu na vzejití.
Při této fázi může dojít k velkým mrazům, které zničí část úrody (v modelu sníží kvalitu úrody\footnote{Kvalitou zrní v modelu reprezentujeme ztrátu úrody zapřičiněnou negativními faktory.})\cite{AgroManualPrezimovani}.
Šance, že taková situace nastane, je v modelu nastavena na 33\,\%\cite{CHMU} a může zničit část úrody danou normálním rozdělením se středem 13 a rozptylem 3.
Před první fází může přijít do hry náš futuristický prvek, a to invaze mimozemské civilizace, která mimo jiné zničí i celou naši úrodu.
To znamená, že produkce mouky daný rok je nulová.
Tato invaze nastane průměrně jednou za 200 let.
Rok po invazi produkce pokračuje jako obvykle. 

V další fázi pšenice přečkala zimu a dalších pět měsíců roste a zraje.
V této fázi úrodu mohou poškodit tři faktory: 
\begin{itemize}
    \item období sucha\cite{AgroManual},
    \item období silných dešťů\cite{AgroManualPoskozeni2021},
    \item škůdci\cite{BittnerSkodliveOrganismy}.
\end{itemize}
Každý z nich může nastat nezávisle na sobě a s pravděpodobností 33\,\% a může zničit část úrody danou normálním rozdělením se středem 13 a rozptylem 3.
Pokud se těchto faktorů v jednom roce objeví více, každý další úbytek kvality pšenice se odečte z již zbylé úrody.

Třetí fáze je sklizeň. Jakmile se do této fáze dostaneme, máme jeden měsíc na sklizení celé úrody.
Auto dokáže pojmout úrodu jednoho hektaru pole, který odpovídá pěti tunám zrní, a poté ji odváží do skladu mlýna.
Řekněme, že cesta auta do skladu i zpět na pole trvá přibližně 20 minut a výklad zrní trvá 10 minut, s tím, že řidič a obsluha skladu mohou začít konverzaci,
která řidiče zdrží na dobu danou exponenciálním rozdělením se středem 3 minuty.

Kombajn pracuje jen pokud má právě k dispozici auto na odvoz sklizeného zrní.
Pokud auto k dispozici nemá, čeká na jeho návrat.
Samotná sklizeň jednoho hektaru trvá kombajnu dobu danou normálním rozdělením se středem 16 minut a rozptylem 1,5 minuty.
S každým dalším hektarem existuje pravděpodobnost, že se kombajn porouchá a nebude v provozu dobu danou exponenciálním rozdělením se středem dva dny.
Na sklízení je vyhrazeno 10 hodin každého dne. 

Po uplynutí měsíce se nesklizená část pole zahodí.

\subsection{Technologie}\label{1.1_technologie}
Model byl implementován a odsimulován v jazyce C++ s pomocí knihovny \href{https://www.fit.vutbr.cz/~peringer/SIMLIB/}{\textcolor{blue}{SIMLIB}}.

\section{Koncepce modelu}\label{2_koncepce}

Futuristický prvek mimozemské civilizace není vzhledem k jeho futuristickému konceptu podložen fakty a pravděpodobnost jeho výskytu jsme stanovili na velmi malou, ale jeho vliv je poměrně velký.

Samotné zpracování zrní v mlýně již není pro náš model podstatné, a tedy z veškerého dostupného zrní je vypočítána hmotnost výsledného produktu, tedy mouky, bez modelování zpracování (čištění, mletí, apod.).
Ve výsledném výpočtu používáme data, která poskytne samotná simulace\,--\,sklizené zrní a jeho kvalita.
Navíc zde pracujeme i s mírou extrakce, kterou jsme stanovili na 70-80\,\%\cite{Alhendi2021}.

Negativní faktory ovlivňující produkci by reálně část úrody zničily, nicméně v našem modelu pouze snižují kvalitu.
Kombajn vždy musí sklidit všech 200\,ha pole, do jednoho auta se vždy vleze jeden hektar a ten odpovídá pěti tunám zrní.

Další zjednodušení modelu spočívá ve vzájemné nezávislosti negativních faktorů.
Například pokud nastane období sucha, pravděpodobnost velkých dešťů se nijak nezmění.
Ale každý další snižuje procentuální kvalitu toho, co zbylo po předchozím.
Například pokud mráz snížil kvalitu o 10\,\%, a poté škůdci sníží kvalitu o dalších 10\,\%, kvalita se sníží dohromady jen o 19\,\%.

Jak bylo zmíněno v předchozí kapitole, celý náš podnik je hypotetický.
Předpokládáme tedy věci typu: auto a kombajn má vždy kdo řídit, mají vždy dost paliva, porucha kombajnu může nastat s~nějakou pravděpodobností, atd..

\section{Implementace}\label{3_implementace}

Simulační program se skládá ze čtyř hlavních procesů, a to \texttt{Quality}, \texttt{Harvester}, \texttt{Harvest} a \texttt{DayCycle}. 

Proces \texttt{Quality} simuluje první dvě fáze, tedy zimu a růst pšenice.
Aktivují se zde procesy poruch (negativní faktor), které se s nějakou pravděpodobností projeví a sníží celkovou kvalitu úrody.

Po dokončení růstu se aktivují procesy \texttt{Harvester} a \texttt{Harvest}.

Proces \texttt{Harvester} simuluje sklizeň pole kombajnem po hektarech. Hektary jsou reprezentovány skladem, ze kterého si kombajn postupně zabírá hektary ke zpracování.
Kombajn k práci potřebuje i auta, která jsou také reprezentována skladem.
Kombajn se tedy pokusí si zabrat auto, na které někdy bude muset čekat z důvodu cesty auta do skladu a zpět.
Pokaždé než si kombajn chce začít zabírat zdroje ke sklizni, existuje 5\% pravděpodobnost, že se porouchá a bude se opravovat dobu danou exponenciálním rozdělením se středem 2 dny.

Proces \texttt{DayCycle} simuluje denní cyklus. 
Nejdříve 10 hodin čeká a poté si na 14 hodin zabere obslužnou linku \texttt{day}, kterou procecs \texttt{Harvester} potřebuje, aby mohl pracovat.
Takhle se opakuje do konce simulačního běhu.

Proces \texttt{Harvest} simuluje dobu sklizně stanovenou na jeden měsíc.
Po uplynutí jednoho měsíce vyprázdní sklad hektarů a tím zastaví proces \texttt{Harvester}, což simuluje situaci, kdy se nestihlo sklidit celé pole.
Zbylé hektary jsou tedy zahozeny.

Statistika se sbírá z:
\begin{itemize}
    \item množství vyprodukované mouky,
    \item kvality pšenice,
    \item počtu oprav kombajnu a průměrného času stráveného opravou,
    \item doby, kterou auto strávilo ve skladu,
    \item počtu hektarů, které se nestihly sklidit během sklizně,
    \item počtu invazí mimozemšťanů.
\end{itemize}
Z těchto statistik pak lze vyhodnotit závěry.

\section{Experimenty}\label{4_experimenty}

Cílem experimentování s modelem je zjistit, jaká je produkce mouky farmy, když růst pšenice, sklizeň a další ovlivňují různé faktory.
Je možné hledat limity, časů a pravděpodobností, pro které bude farma schopna zpracovat veškerou úrodu.
Příkladem může být různý rozsah denní pracovní doby, doba zdržení auta ve skladu zrní, doba oprav kombajnu, pravděpodobnost poruch kombajnu, a podobně.

\subsection{Postup experimentování}\label{4_1_postup_exp}

V kódu jsou definovány konstanty, které se dají libovolně modifikovat.
Dá se modifikovat doba zdržení auta ve skladu, kolik auto uveze tun nebo doba sklízení jednoho hektaru, čímž se dá simulovat různá výkonnost kombajnu a různé další konstanty.

Dají se odsimulovat i jiné podnebné podmínky změnou pravděpodobností výskytu a mírou vlivu negativních faktorů spojených s počasím.

Výstup z experimentů se dá použít například k\ldots

\subsection{Příklady experimentů}\label{4_2_priklady}

Experimenty, které byly na modelu provedeny, se týkaly klimatických změn, oprav kombajnu a doby vyložení auta ve skladu.
V každém experimentu je simulace spuštěna 10000$\times$.

\subsubsection{Změna klimatu}\label{4_2_1_zmena}

Pro zvýšení validace modelu byl jako první proveden experiment, kdy tuhé mrazy přišly každou zimu a vždy zničily 100\,\% úrody.
Očekávaným, a zároveň skutečným, výsledkem byla výsledná produkce mouky 0 tun za rok. 

Prvním experimentem byl pokus o simulaci tužších klimatických podmínek.
Výchozí hodnoty pro pravděpodobnost tuhé zimy a pro úbytek úrody jsou 33\% a 13\%.

Nejprve v první tabulce můžeme vidět výsledky experimentu, ve kterém jsme zvyšovali prav\-dě\-po\-dob\-nost příchodu tuhé zimy a sledovali, jak ovlivňuje produkci.

\begin{table}[ht]
    \centering
    \begin{tabularx}{\textwidth}{|p{0.4\textwidth}|X|X|}
        \hline
        \textbf{Pravděpodobnost tuhé zimy} & \textbf{Kvalita v \%} & \textbf{Mouka v t} \\
        \hline
        33\% & 83 & 624 \\
        \hline
        40\% & 83 & 620 \\
        \hline
        50\% & 82 &  610 \\
        \hline
        100\% & 76 & 571 \\
        \hline
    \end{tabularx}
    \caption{Závislost produkce na zvyšující se pravděpodobnosti příchodu tuhé zimy.}
    \label{tab:zima_casto}
\end{table}

\newpage
Ve druhé tabulce jsou výsledky experimentu, kde se zvyšuje tuhost zimy, e.g. procento úbytku úrody při příchodu tuhé zimy. 

\begin{table}[ht]
    \centering
    \begin{tabularx}{\textwidth}{|X|X|X|}
      \hline
      \textbf{\% úbytku úrody} & \textbf{Kvalita v \%} & \textbf{Mouka v t} \\
      \hline
      13 & 83 & 624 \\
      \hline
      15 & 83 & 621 \\
      \hline
      30 & 79 & 589 \\
      \hline
      50 & 72 & 540 \\
      \hline
    \end{tabularx}
    \caption{Závislost produkce na zvyšujícím se úbytku úrody při příchodu tuhé zimy.}
    \label{tab:zima_ubytek}
\end{table}
  
V poslední tabulce tohoto experimentu sledujeme reálné tužší podnebí.
To znamená častější a tužší zimy spojené s větší deštivostí v létě.
Zkratky: \emph{PTZ} - pravděpodobnost příchodu tuhé zimy, \emph{TZ} - tuhost zimy, úbytek úrody v tuhé zimě,
\emph{PD} - pravděpodobnost silných dešťů, \emph{ND} - ničivost dešťů, \emph{K} - kvalita, \emph{M} - mouka.

\begin{table}[ht]
    \centering
    \begin{tabularx}{\textwidth}{|X|X|X|X|X|X|}
      \hline
      \textbf{PTZ\%} & \textbf{TZ\%} & \textbf{PD\%} & \textbf{ND\%} & \textbf{K\%}  & \textbf{Mt} \\
      \hline
      33 & 13 & 33 & 13 & 83 & 624\\
      \hline
      40 & 30 & 40 & 30 & 71 & 530\\
      \hline
      50 & 30 & 50 & 30 & 66 & 494\\
      \hline
      50 & 50 & 50 & 50 & 51 & 383\\
      \hline
    \end{tabularx}
    \caption{Závislost produkce na tužších a častější zimách a silnějších a častějších deštích.}
    \label{tab:example}
\end{table}


\subsubsection{Řidičova výřečnost}\label{4_2_2_vykecavani}

Dalším experimentem byla simulace výřečnějších řidičů.
Když řidič odváží náklad do skladu, může se tam zdržet v rámci přátelských konverzací.
Výchozí doba strávená ve skladu je 10 minut (výklad zrní) a~navíc doba daná exponenciálním rozdělením se středem 3 minuty.

\begin{table}[ht]
    \centering
    \begin{tabularx}{\textwidth}{|X|X|}
      \hline
      \textbf{Střední hodnota zdržení v minutách} & \textbf{Počet nesklizených hektarů pole} \\
      \hline
      3 & 2,3 \\
      \hline
      4 & 2,5 \\
      \hline
      10 & 2,8 \\
      \hline
      20 & 3,5 \\
      \hline
      50 & 6,3 \\
      \hline
    \end{tabularx}
    \caption{Závislost nesklizených hektarů pole na výřečnosti řidiče.}
    \label{tab:vykecavani}
\end{table}

\subsubsection{Lakatoš}\label{4_2_3_lakatos}

Experiment s pracovním názvem Lakatoš spočíval v simulaci větší pravděpodobnosti poruchy kombajnu a delší doby opravování.
Výchozí doba opravy je daná exponenciálním rozdělením se středem 2 dny.

V tabulce 5 sledujeme, jak se mění počet nesklizených hektarů, když se zvyšuje trvání opravy kombajnu.
V tabulce 6 sledujeme, jak se mění počet nesklizených hektarů, když se zvyšuje pravděpodobnost poruchy kombajnu.

\newpage

\begin{table}[ht]
    \centering
    \begin{tabularx}{\textwidth}{|X|X|}
      \hline
      \textbf{Střední hodnota doby opravy ve dnech} & \textbf{Počet nesklizených hektarů pole} \\
      \hline
      2 & 2,3 \\
      \hline
      3 & 9,1 \\
      \hline
      5 & 29,1 \\
      \hline
      10 & 68,4 \\
      \hline
      20 & 102,7 \\
      \hline
    \end{tabularx}
    \caption{Závislost nesklizených hektarů pole na době opravy kombajnu.}
    \label{tab:lakatos1}
\end{table}

\begin{table}[ht]
    \centering
    \begin{tabularx}{\textwidth}{|X|X|}
      \hline
      \textbf{Pravděpodobnost poruchy v \%} & \textbf{Počet nesklizených hektarů pole} \\
      \hline
      2 & 2,3 \\
      \hline
      3 & 6 \\
      \hline
      5 & 22,5 \\
      \hline
      10 & 80,1 \\
      \hline
      20 & 133 \\
      \hline
    \end{tabularx}
    \caption{Závislost nesklizených hektarů pole na pravděpodobnosti poruchy kombajnu.}
    \label{tab:lakatosě}
\end{table}

\subsection{Odhalené chyby}\label{4_3_chyby}

Během experimentování s vytvořeným modelem byly odhaleny chyby v implementaci.
Mezi chyby patří neaktivování procesu denního cyklu, což způsobilo vždy kompletní sklizeň, odhalení chybného plánování procesu, kdy při 100\% pravděpodobnosti příchodu zimy a zároveň 100\% ničivosti jsme stále viděli nenulovou produkci mouky. 

\section{Závěr}\label{5_zaver}

Experimenty prováděné na modelu tvrdí, že pěstování pšenice v drsnějších klimatických podmínkách vede na nižší produkci mouky, nicméně stále je to reálné, počítáme-li s touto skutečností.

Dále tvrdí, že i s řidiči, kteří se déle zdržují ve skladu a zdržují tak proces sklizně, je možné sklizeň stihnout včas.
I když se zdrží drasticky déle, celkovou úspěšnost sklizně tolik neovlivní.

Z experimentu Lakatoš vyplývá, že spolehlivost kombajnu je mnohem kritičtější a vyšší poruchovost či doba opravy značně ovlivňuje plynulý průběh sklizně.

Na tomto modelu se dá zkoumat vliv různých faktorů negativně ovlivňujích produkci mouky hypotetického podniku.
Dokazují to experimenty provedené v rámci této studie.

\newpage

\bibliography{references}

\end{document}